% \subsubsection*{What is Machine Ethics ?}
Can machines act ethically?\ That is what machine ethics, a newly emerging field, grown at the intersection of philosophical research and research in engineering and computer science, studies the creation of “ethical machines.” Such machines follow an ideal set of principles and, guided by these principles, decide about possible courses of action.\ We argue that someday machines will become good ethical decision-makers, at least in a limited sense, by appealing to ethical principles, and also act according to them.

% \subsubsection*{Importance of Machine Ethics:}
\indent Anderson et al.\ \cite{Anderson_Anderson_2007} have argued about the importance of machine ethics.\ For instance, success in DARPA’s (Defense Advanced Research Projects Agency) grand challenge has paved a way for the development of self-driving vehicles that are now manoeuvring in an urban environment.\ Such vehicles have been studied in regards to, e.g., malicious hacking, inevitable crash optimization and others that question their reliability and safety.\ Another example, the United States Army’s Future Combat Systems program has been developing armed robotic vehicles that will support ground troops with “direct-fire” and antitank weapons -- which will decide who lives and who does not.\ Such machines will be capable of causing harm to human beings and have ethical ramifications.\ So that they treat us well, adding an ethical component is important.\ Second, machines are becoming more autonomous.\ The more powerful the AI we incorporate into them, the more powerful machine ethics is needed.\ Further, research in machine ethics can advance ethics in general.\ Thus, a third reason is programming and teaching a machine to act ethically can lead to the development of better domain ethical theories.

% \subsubsection*{Types of Ethical Machines:}
\indent In an oft-cited paper \cite{1667948}, Moor defines four different levels of ethical agents.\ Ethical impact agents are those whose actions have ethical consequences whether intended or not e.g., a digital clock has the consequence of encouraging its user to be on time for an appointment.\ Implicit ethical agents have ethical considerations built into (implicit in) their design.\ Often, these are safety and security considerations, e.g., Automatic Teller Machines (ATM).\ They must give out the right amount of money, must check the availability of funds, and often limit the amount of daily withdrawals, i.e., these agents have designed reflexes for situations that require monitoring to ensure security.\ Explicit ethical agents identify and process ethical information, make decisions about what they should do, and reason about ethical principles when in conflict.\ Full ethical agents make ethical judgements about a wide variety of situations and can justify like explicit ones, along with having features like consciousness, intentionality, and free will.

% \subsubsection*{Why focus on Explicit Ethical Machines ?}
\indent The key finding of this report is that implementing explicit ethical agents using a hybrid way of computing, discussed in Section 2, is a possible approach to make machine ethics philosophically interesting.\ The advantages of explicitly representing ethical principles are: first, it allows the ability to justify judgments by appealing to them.\ Second, it allows providing transparency to humans beings who will question their competency in new situations.\ Lastly, it gives an advantage over human beings, who despite being taught ethics tend to sometimes favour themselves, which the machine cannot override.\ However, challenges posed by this hybrid methodology are numerous, as discussed in Section 5, and only working on them will help us gain clarity about the machine's behaviour.